\documentclass[12pt]{article}
\usepackage{caption}
\usepackage{subfloat}
\usepackage[digitsep={,}]{siunitx}
\usepackage[margin=1in]{geometry}
\usepackage{amsmath,amsthm,amssymb,bbm,booktabs,graphicx,xcolor,pdflscape,tabularx,threeparttable,rotating,lscape}
\usepackage[round,authoryear]{natbib}
%\bibliographystyle{plainnat}
\usepackage[bottom]{footmisc}
%\usepackage[capposition=top]{floatrow}
\usepackage[hidelinks]{hyperref}
\usepackage{color}
\definecolor{webgreen}{rgb}{0,.5,0}
\definecolor{webbrown}{rgb}{.6,0,0}
\definecolor{webpurple}{rgb}{0.7,0,0.7}
\usepackage{tgtermes}
\usepackage{hyperref}
\hypersetup{breaklinks = true,  letterpaper=false, colorlinks=true, anchorcolor= webbrown, citecolor= webbrown, filecolor= webbrown, linkcolor= webbrown, menucolor= webbrown, urlcolor= webbrown, citebordercolor= 1 0 0, menubordercolor=1 0 0, urlbordercolor=1 0 0, runbordercolor=1 0 0 }
\usepackage{setspace}
\usepackage{comment}
\usepackage{epstopdf}
\usepackage{float}
\usepackage{tikz}
\usepackage{stackengine}
\stackMath
\usepackage{titlesec}
\usepackage{caption}
\setlength{\belowcaptionskip}{-5pt}
\usepackage{setspace}
\usepackage{placeins}
\usepackage{etoolbox}
\usepackage{adjustbox}
\usepackage{multirow}
\AtBeginEnvironment{quote}{\singlespacing\small}

% Allow line breaks with \\ in specialcells
	\newcommand{\specialcell}[2][c]{%
	\begin{tabular}[#1]{@{}c@{}}#2\end{tabular}}

\DeclareCaptionType{mytype}[Typename][List of mytype]
\newenvironment{myenv}{\captionsetup{type=mytype}}{}


\begin{document}
%% The key part is here, we import all the numbers we will call.
\newcommand{\statA}{12345.678}
\newcommand{\statB}{12345.68}
\newcommand{\statC}{12,346}
\newcommand{\statD}{12345.678} % This is a comment. Look at line 25 of fillnumbers.do.
\newcommand{\statE}{12345.678} % Created by stattotex at 11:16:20 on 28 Mar 2020.
\newcommand{\statF}{12345.678} % This is a comment. Created by stattotex at 11:16:20 on 28 Mar 2020.
\newcommand{\meanZ}{2.04} % This is the mean Z when X>0.6 & Y<=0. Created by stattotex at 11:16:20 on 28 Mar 2020.


\begin{titlepage} 

\title{\Large This is the title
\thanks{Thank you to....}\vspace{1cm}}

\author{Author's Name \\ Author's Institution \vspace{1cm}}
\date{March 2020}
\maketitle
\thispagestyle{empty}
\bigskip

\centerline{\bf Abstract}\vspace{0.5cm}

\noindent The abstract goes here

\end{titlepage}

\doublespace
\onehalfspacing

\section{Introduction}

This template shows how to use latex and Stata to automatically fill in numbers that are placed in the text. This allows us to never hard-code any numbers, which is more efficient and reduces the risk of making mistakes.

The key file is \textsf{numbersintext.tex} -- see the do file \textsf{fillnumbers.do} for how this is created. This file is populated with commands that take the form \textsf{\textbackslash newcommand\{\textbackslash NAME\}\{STATISTIC\}}, so that whenever you call \textbackslash NAME, STATISTIC will show up.

All you need to do to achieve this is place this at the top of your tex file: \textsf{\textbackslash input\{numbersintext.tex\}}.

Then you can call all the variables you created. For example, this is the value of \textsf{\textbackslash statA}: $\statA$. This is \textsf{\textbackslash statC}: $\statC$. And this is \textsf{\textbackslash meanZ}: \meanZ.



\end{document}